\documentclass[12pt]{report}
\author{Joshua Millikan \\ Alex Eckert\\ Dominic Turmenne \\ Franque Gonzalez}
\usepackage{listings}
\usepackage{underscore}
\usepackage{graphicx}
\usepackage[bookmarks=true]{hyperref}
\usepackage[utf8]{inputenc}
\usepackage{titlesec}
\usepackage[acronym]{glossaries}
\usepackage[scaled]{helvet}
%\usepackage{mathptmx}
\usepackage[T1]{fontenc}
\renewcommand\familydefault{\sfdefault}

\title{Magic the Gathering Deck Builder}
\titleformat{\chapter}[display]
{\normalfont\huge\bfseries}{}{0pt}{\Huge}
\titlespacing*{\chapter}
{0pt}{10pt}{40pt}
\hypersetup{
	bookmarks=false,
	colorlinks=true,      
	linkcolor=black,      
	citecolor=black,       
	filecolor=black,      
	urlcolor=blue,       
}

\newcommand{\javaversion}{10 }

\setglossarysection{section}

\makenoidxglossaries 

\newacronym{jvm}{JVM}{Java virtual machine}

\newacronym{mtg}{MTG}{Magic the Gathering}

\newacronym{api}{API}{application programming interface}

\newacronym{tls}{TLS}{transport layer security}

\newglossaryentry{user}{name={user},description={the person using this software for it's intended purpose}}

\newglossaryentry{https}{name={HTTPS},description={an extension of Hyper Text Transfer Protocol(HTTP) that is encrypted using transport layer security}}

\newglossaryentry{rest}{name={REST},description={Respresentation state transfer, a HTTP based web service protocol}}

\newglossaryentry{magic}{name={Magic the Gathering},description={A card game first released by Wizards of the Coast LLC in 1993}}

\newglossaryentry{sqlite}{name={SQLite},description={A embedded SQL database engine that does not run as a seprate server process and writes directly to an ordinary file}}

\newenvironment{unbroken}
{\par\nobreak\vfil\penalty0\vfilneg
	\vtop\bgroup}
{\par\xdef\tpd{\the\prevdepth}\egroup
	\prevdepth=\tpd}

\widowpenalties 1 10000
\raggedbottom

\begin{document}
	\fontfamily{phv}
	\makeatletter
	\begin{titlepage}
		\centering
		\huge{\textbf{\@title}}
		\large{\\ \textbf{Software Requirements Documentation}}\\
		\vspace{1.5cm}
		Player 4\\
		\today\\
		\vspace{1.5cm}
		\textbf{Authors:}\\\@author\\
		\vspace{\fill}
		\small{\textit{We have abided by the UNCG Academic Integrity Policy on this Assignment}}\\
		\hfill\\
		\rule{16cm}{5pt}\vskip1cm
		
	\end{titlepage}
\tableofcontents
%1
\chapter{Introduction}
\section{Purpose}
This application is intended to assist the \gls{user} with managing decks of cards for the \gls{magic} card game.
\section{Document Conventions}
This document was created as a group effort by all members of our group, and is formatted using \LaTeX. all text is in Computer Modern 12pt font.
\section{Intended Audience}
This document is intended for our development team to provide guidance in developing this application, and our professor, to grade as part of this project.
\clearpage
\section{Definitions}
\printnoidxglossaries   
\section{Project Scope}
This application will run on locally on the \gls{user}'s computer and will allow the user to create lists of cards representing a deck using up to date information accessed from the internet. The user will be able to save the deck to a local database as well as load and modify previously saved decks. This application will provide relevant information to the user and enforce game rules for what can be put in a deck, including rules such as no more than 4 cards with the same name in a deck, and not allowing cards that are officially banned in the chosen format. The user will have a variety of options including the ability to select what format of \gls{magic} to use the rules for, as well as searching and filtering cards based on attributes such as name, mana color, release set, power, toughness, and card type (creature, land, etc..).
\section{Technical Challenges}
For most of the developers, this is their first time working on a large programming project with multiple people, which may make organization and coordination a challenge.
\section{References}
\acrshort{api} used: \url{https://scryfall.com/docs/api}
%2
\chapter{Overall Description}
\section{Product Features}
The proposed project is a deck building application for the collectible card game “\Gls{magic}”.  \acrshort{mtg} is an exciting game where players build a variety of decks from thousands of card choices. While most players collect and play the game using physical cards, the sheer amount of distinct cards in existence precludes them from being able view every card, and a better system is needed for identifying cards relevant to their current deck. Our project seeks to eliminate this problem by creating a program that allows users to search for cards using an existing \acrshort{api} database. \Gls{user} will be able to search using a variety of parameters and restrictions. As they find cards of interest, they will then be able to add them to one or multiple decks. Options to remove cards from existing decks will also be included. Users will be able to save their work, and return at any time.
\section{\Gls{user} Characteristics}
Our target audience is MTG players of all ages. Although this encompasses a wide range of ages, users will skew in age towards teenagers, and young adults. Despite having no specific training, users are assumed to have a general knowledge of computer usage. They are likely to use our product at home for recreational purposes.
\section{Operating Environment}
The proposed project will be deployed as a desktop application; user input will be received via a combination of key, and mouse strokes. Should be able to run on any system with Java version \javaversion or greater installed. System requirements should be kept low to improve accessibility; the final product should require no more than RAM of 2GB and 100 MB of disk space.
\section{Design and Implementation Constraints}
Implementation will be completed using NetBeans/Intellij to build a suitable JFrame architecture. Ease of use is paramount, as many users are not technologically advanced, and will use the program recreationally. All card data needs to be pulled from an existing external database using API. All Deck data will be stored using a localized database. Excluding a card name, No other data on cards will be stored locally. Users need to be able browse and delete from their deck, as well as search and add new cards. Both of these functions will require comparing the card name in the local database, against the matching card name in the \acrshort{api} database. There also should be a feature that checks the validity of a card for a chosen rule set.  Adapters should be implemented to support future deployments.
\section{Assumptions and Dependencies}
The application is highly dependent on an external \acrshort{api}. It is assumed this database will continue to be available and accessible, without major changes in structure. Searching, and adding cards is entirely dependent on the external database, and will not work should that connection be broken. Displaying decks will likewise be unable to display the full details; however, in the interests of usability, users should still be able to retrieve cards in their existing decks using the localized database. The deletion of cards from decks is completely independent of the \acrshort{api}. The application's database is planned to be implemented using \gls{sqlite}, however it should not be strongly dependent on any particular database and should be designed to make it easy to change what kind of database is used if necessary.
%3
\chapter{Functional Requirements}
\section{Primary}
Permanent data storage is required for saving deck information. Users need to be able to add and remove cards, as well as decks. There should also be checks in place to test the tournament validity of each card as it is added to a deck. The application must be able gather information from the user, then search the \acrshort{api} database, and return relevant card results based on the given input. The software needs to present both deck results, and search results in an easy to browse format. On one page, a listing of all relevant card names should be viewable, as well as an active card display showing the complete details of that card.
\section{Secondary}
Ideally, the app should include features that allow easy retrieval of card names in the event the \acrshort{api} connection is severed. Back end design should be conducive towards future expansion. A built in printer-formatting feature would also improve user experience. 
%4
\chapter{Technical Requirements}
\section{Operating Systems/Compatibility}
The Magic of The Gathering Deck Builder application will run on the \acrshort{jvm} and as such should be compatible with any system with Java installed. This application will be built using JDK version \javaversion and as such with require a Java runtime version of \javaversion or greater to run.
\section{Interface Requirements}
\subsection{User Interface}
The application includes many user interface tools. Those include a look up new cards feature, to see which cards are available to be added to the deck. This feature will call upon the \acrshort{api} to see all the possible cards and look them up based on their name. Also, a part of the user interface includes an add card button which will allow the user to add any card that they look up to the deck of their choice. A remove card feature will be implemented to remove any card from whichever deck is chosen. A feature to display the deck will also be available to see all the cards that are present within a deck. There will be a feature to start a new deck, this will allow multiple decks to be made with different names and attributes. Finally, a remove deck feature will be implemented to allow the user to remove a deck that the user no longer wishes to keep.
\subsection{Hardware Interface}
The application will not implement many uses of hardware. The only hardware prevalent will be the inputs of the keyboard and mouse.
\subsection{Software Interface}
The application will be utilizing an \acrshort{api} in order to retrieve data from the \gls{magic} game in order to find cards and figure out what they do so that they may be used in the deck building application. The \acrshort{api} being used is provided by Scryfall. It contains a \gls{rest}-like \acrshort{api} for ingesting card data programmatically. The \acrshort{api} exposes information available on the regular site in easy-to-consume formats. The \acrshort{api} uses UTF-8 character encoding for all responses. Also, \acrshort{api} requests are only served over \gls{https}, using \acrshort{tls} 1.0, 1.1, and 1.2.
\subsection{Communications Interface}
The application will not utilize any communications interface.
%5
\chapter{Nonfunctional Requirements}
\section{Performance Requirements}
This application is planned to lack intensity on the user’s system. This application should be capable of running on any computer capable of running Java \javaversion or newer.
\section{Safety/Recovery Requirements}
This application will periodically save the user’s data (created decks, cards, etc.) 
to a localized database during use. In the event that the application 
experiences an unexpected crash or shutdown, the user’s most recently saved
data will be reloaded from the local database upon starting the application.
If the user’s storage system becomes entirely corrupted or compromised, there 
will be no cloud-based saved data to pull from, making their data effectively unrecoverable. For this reason, it is important that the program maintains its high integrity.
\section{Security Requirements}
While using this application, the user will not have to log into or create an 
account to access the application’s features of searching cards and creating/managing decks.

\section{Policy Requirements}
As per our policy requirements, this application will adhere to the policies of the \acrshort{api} in regards to the handling of user data and to the validity of cards in the rule-sets as specified by \acrshort{mtg}.
\section{Software Quality Attributes}
\subsection{Availability}
The service this application provides is available on demand by user request.
\subsection{Correctness}
This application will aim to accurately display card information with their most recently updated values and their tournament legality given by the implemented \acrshort{api}.
\subsection{Maintainability}
This application will not be updated regularly outside of the necessary 
adjustments if and when the \acrshort{mtg} \acrshort{api} framework is updated to support/change data fields.
\subsection{Reusability}
All of the application’s code elements will not be used as a part of another 
application with the exception of the \acrshort{api} connection code possibly being 
used at a later date.
\subsection{Portability}
This application is strictly designed for desktop systems that can run java applications. As it is implemented in Java, the application should be able to run on any system with Java \javaversion or greater installed
\begin{unbroken}
\section{Process Requirements}
\subsection{Development Process Used}
During this application’s development, a waterfall implementation process was used. The project team was divided into four roles in charge of a 
specific task:
\begin{enumerate}
	\item Github administrator: Manages Github pull requests and branch merging.
	\item UI/Frontend Developer: Develops the user interface from which the user interacts with our application.
	\item Persistent Data Storage: Manages the storage of the user’s data in an effective manner.
	\item API Connection: Connects the application to the \acrshort{mtg} \acrshort{api} and parses necessary information.
\end{enumerate}
\end{unbroken}
\subsection{Time Constraints}
The project team was given four months (January - April) to build this application in its entirety.
\subsection{Cost and Delivery Date}
As is standard with project-integrated-courses, the team will build this 
application during the Spring 2021 UNCG semester without pay and is to be delivered by the end of the semester on April 30th, 2021. 

\end{document}